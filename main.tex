\documentclass{article}
\usepackage[utf8]{inputenc}
\usepackage{graphicx}
\usepackage{algpseudocode}
\usepackage{algorithm}
\begin{document}

\begin{figure}
  \includegraphics[width=\linewidth]{output_55_0.png}
  \caption{cumulative counts.}
  \label{fig:cumulative counts}
\end{figure}

Figure \ref{fig:cumulative counts} contoh untuk mengambil rerefensi.

\begin{table}[h!]
\center
\begin{center}
 \begin{tabular}{||c c c||} 
 \hline
 Method & Mean error & Mean absolute error \\ [1ex]
 \hline\hline
 Baseline & $-32.3\mp3.9$ & $32.3\mp3.9$ \\  
 Meal Snap & $-32.3\mp3.9$ & $32.3\mp3.9$ \\
 Menu Match & $-32.3\mp3.9$ & $32.3\mp3.9$ \\
 C & $-32.3\mp3.9$ & $32.3\mp3.9$ \\
 \hline
\end{tabular}
\end{center}
\caption{Contoh Tabel}
\label{table:1}
\end{table}

\begin{algorithm}
  \caption{Algorithm for finding server indices using OFG}

  \begin{algorithmic}
    \Statex \Comment { \%comment: servers[] contains the index of servers whose         data rate are sorted in descending order\%}
    \State servers[]= index(of all servers) 
    \State serverIndex[]=servers[0..K]
    \State linearlyIndependentServerIndex[]=0
    \State $[Z] \leftarrow 0$
    \For  {$i=0$ to $serverIndex.length$} 
    \Statex\Comment{ \%comment: find the equation corresponding the serverIndex        from the mapping at the File Server\%} 
    \State        $eqn= equation(serverIndex[i])$ 
    \Statex\Comment{ \%comment: try insert equation into Z using OFG\%} 

    \EndFor end for 
    \While{ ( linearlyIndependentServerIndex.length!=K ) } 
    \Statex\Comment{\%comment: remove all the server index which were not inserted in Z\%}  
    \State temp[]=serverIndex[]-linearlyIndependentServerIndex 
    \If{  (linearlyIndependentServerIndex.length=K) }
    \State break
    \EndIf  
    \EndWhile  
  \end{algorithmic}
\end{algorithm}

\end{document}
